\documentclass[a4paper,10pt]{report}
\usepackage[utf8]{inputenc}
\usepackage{cancel}
\usepackage[margin=2cm]{geometry}
\usepackage{amsmath}
\usepackage{amssymb}

% Title Page
\title{Mathematical methods}
\author{Lachlan Takumi Ikeguchi}

\begin{document}
\maketitle
\tableofcontents

\begin{abstract}
	This document was written to be used as a summary to help revise the content covered mathematical methods.  For any inquiries, feedback, and further explanations, contact lachlanprivate@duck.com or through the discord server: https://discord.gg/6P8rddkXFr
\end{abstract}

\section{Important symbols}
\begin{center}
	\begin{tabular}{l|lp{6cm}}
		Symbol & Mathematical definition          & Simple definition                               \\ \hline
		$\cup$ & $P(A \cup B) = P(A) + P(B)$      & Probability of A occuring \emph{or} B occuring  \\
		$\cap$ & $P(A \cap B) = P(A) \times P(B)$ & Probability of A occuring \emph{and} B occuring \\
	\end{tabular}
\end{center}

\pagebreak

\section{Arithmetic sequence}
An arithmetic sequence is a sequence of numbers where the terms are increasing or decreasing at a constant rate.  The recursive definition of a sequence is:
$$t_{n + 1} = t_n + d$$
where $t_{n + 1}$ is the next term, $t_n$ is the current term, and $d$ is the common difference or ``by how much the sequence goes up by"\\

And the general equation for finding the $n^{\text{th}}$ term of the sequence is:
$$t_n = t_1 + (n - 1)d$$
$t_n$ is the $n^{\text{th}}$ term of the sequence, $t_1$ is the first term of the sequence, and $d$ is the common difference.

\subsection{Sum of all prior arithmetic sequence}
The sum of all arithmetic sequence is given by:
$$S_n = \frac{n}{2}(2t_1 + (n - 1)d) = \frac{n}{2}(t_1 + t_n)$$

\section{Functions}
Functions is an mathematical concept where if given an input, it performs some operations, and returns an output.  They have the mathematical notation of $f(x)$ where $f$ can be seen as the name of the function and the values within the parenthesis as inputs to the function.  A function must be defined for it to be used.  A simple function which calculates the y-values along a parabola with the x-coordinates as input could be:
$$M(x) = x^2$$

This function would return the values:
\begin{center}
	\begin{tabular}{l|l}
		$x$ & $M(x)$ \\ \hline
		-3  & 9      \\
		-2  & 4      \\
		-1  & 1      \\
		0   & 0      \\
		1   & 1      \\
		2   & 4      \\
		3   & 9
	\end{tabular}
\end{center}

Functions can only give an one-to-one relation or a many-to-one relation since there is a clear defined input and output values with the concept of a function.  To check if a graph is a function a \emph{vertical line test} can be performed:\\

If the graph intersects any vertical line more than once, it cannot be a function, and is instead a relation.

\section{Interval notation}
Interval notation are used to define the range of a function or a relation.  For example, if given a function where the y-values start at 0 and approach but never reach 1, to define the range of the function, it would be $0 \leq y < 1$.  However, this can be rewritten using interval notation as: [0, 1).  The square bracket meaning that it is equal to and less than or greater than, and the parenthesis meaning that it is greater than or less than but never reaching the value.\\

\begin{center}
	\emph{NOTE:  it is called a ``range" when it is the y-value, and a ``domain" when it is the x-value}
\end{center}

\section{Relations and Functions}
There are four types of relations, relations meaning for any x-value, how many valid y-values there are.  The four are: one-to-one, many-to-one, one-to-many, and many-to-many.\\

The first, (one-to-one) relation means that for any x-values, there is only one y-value.  An example would be a straight line: $f(x) = 2x + 3$.  The second, (many-to-one) relation means that for any y-values, there are more than one x-value.  An example would be a parabola: $f(x) = x^2$.  The previous are examples of functions, the next two are not.  The third, (one-to-many) relations means that for any x-value, there are more than one y-value.  And for the last, (many-to-many) relations means that for any x-value, there are more than one y-value, and the reverse is also true.

\section{Transformations}
The ways a function can be transformed by some units are:
$$
	y = a \times f(b(x - c)) + d
$$
Where $a$ is the dialation from the x-axis, $b$ is the dialation from the y-axis, $c$ is the translation along the x-axis, and $d$ is the translation along the y-axis.

\subsection{Dialations}
A dialation transformation means that the function has been either stretched or compressed.  These transformations occur when an output has been multiplied by a dialation factor.  Dialations can either occur in the x-axis or the y-axis.\\
Let y = f(x)\\
When dialated from the x-axis by $a$:
$$
	y = a \times f(x)
$$
When dialated from the y-axis by $b$:
$$
	y = f(x \times b)
$$
\begin{center}
	\emph{NOTE:  values less than 1 compresses the function and greater than 1 stretch it}
\end{center}

\subsection{Reflexions}
A reflexion transformation means that the function has been flipped along an axis.\\
To reflect a function along the x-axis:
$$
	y = -f(x)
$$
To reflect a function along the y-axis:
$$
	y = f(-x)
$$

\subsection{Translations}
A translation transformation means that the function has been shifted along an axis.\\
To move the function along the x-axis by $c$ units:
$$
	y = f(x - c)
$$
To move the function along the y-axis by $d$ units:
$$
	y = f(x) + d
$$

\section{Peicewise functions}
A peicewise function is a function that changes givien some condition:\\
As an example:
$$
	y = \begin{cases}
		x  & x \geq 0 \\
		-x & x < 0
	\end{cases}
$$
This means as $x$ is equal to or greater than 0, $y$ is $x$, but if $x$ is less than 0, $y$ is $-x$

\section{Quadratics}
\subsection{general or polynomial form}
The general or polynomial form of a quadratic is:
$$
	y = ax^2 + bx + c
$$

The axis of symetry or the turning point is found by:
$$
	x = -\frac{b}{2a}
$$

\subsection{Turning point form}
The turning point form is:
$$
	y = a(x - b)^2 + c
$$
Where $b$ is the translation along the x-axis and $c$ is the translation along the y-axis.

\subsection{Factorised or x-intercept form}
The factorised or x-intercept form is:
$$
	y = a(x - b)(x - c)
$$
Where it intercepts the x-axis at points $x = b$ and $x = c$

And the axis of symetry for this form is:
$$
	x = \frac{b + c}{2}
$$

\subsection{Equations from graphs}
If given the turning point, use the turning point form and substitute the coordinates into $b$ and $c$.  And if given the x-intercepts, use the x-intercept form and subtitute the values $b$ and $c$.  While if given 3 points along the curve, use the general form by:\\
\begin{align}
	\text{Given 3 points:} (0, 11), (1, 5), (2, 3)        \\
	\text{Subtitute $x$ and $y$ values into general form} \\
	11 & = a(0)^2 + b(0) + c                              \\
	5  & = a(1)^2 + b(1) + c                              \\
	3  & = a(2)^2 + b(2) + c                              \\
	\text{Solve for unknowns}                             \\
	11 & = \cancel{a(0)^2}{0} + \cancel{b(0)}{0} + c\\
    c & = 11
\end{align}

\subsection{Translations with quadratics}

\subsection{Modelling with quadratics}

\section{Factorizing}
\subsection{Perfect squares}

\subsection{Difference between squares}

\subsection{The discriminant}

\section{Inverse proportion}

\section{Circles}

\section{Operations on polynomials}
\subsection{Division}

\subsection{Remainder theorem}

\subsection{Factor theorem}

\section{Probabilities}

\section{Combinations}

\section{Pascal's triangle}

\section{Geometric sequences}

\end{document}




